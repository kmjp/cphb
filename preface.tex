\begin{comment}
\chapter*{Preface}
\markboth{\MakeUppercase{Preface}}{}
\addcontentsline{toc}{chapter}{Preface}
\end{comment}
\chapter*{????}
\markboth{\MakeUppercase{????}}{}
\addcontentsline{toc}{chapter}{????}

\begin{comment}
The purpose of this book is to give you
a thorough introduction to competitive programming.
It is assumed that you already
know the basics of programming, but no previous
background in competitive programming is needed.
\end{comment}

この本の目的は、読者に競技プログラミングの徹底的な入門書と
なることである。
前提として、読者はプログラミングの基礎知識を持つものとするが、
競技プログラミングのバックグラウンドは必要としない。

\begin{comment}
The book is especially intended for
students who want to learn algorithms and
possibly participate in
the International Olympiad in Informatics (IOI) or
in the International Collegiate Programming Contest (ICPC).
Of course, the book is also suitable for 
anybody else interested in competitive programming.
\end{comment}

この本は特にアルゴリズムを学び国際情報オリンピック(IOI)や
ACM国際大学対抗プログラミングコンテスト(ICPC)への参加したいと
考えている学生向けである。
もちろん競技プログラミングに興味があれば誰にでも適している。

\begin{comment}
It takes a long time to become a good competitive
programmer, but it is also an opportunity to learn a lot.
You can be sure that you will get
a good general understanding of algorithms
if you spend time reading the book,
solving problems and taking part in contests.
\end{comment}

良い競技プログラマになるには長い時間を要するが、
それは多くの学びの機会でもある。
読者はこの本を読み、問題を解き、コンテストに出る時間を通じて
アルゴリズムをより良く理解できるようになるだろう。

\begin{comment}
The book is under continuous development.
You can always send feedback on the book to
\texttt{ahslaaks@cs.helsinki.fi}.
\end{comment}

本書は継続的に進展している最中である。
この本へのフィードバックは\texttt{ahslaaks@cs.helsinki.fi}まで送ってほしい。


\begin{flushright}
Helsinki, October 2017 \\
Antti Laaksonen
\end{flushright}
